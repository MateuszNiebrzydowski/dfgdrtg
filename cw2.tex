\documentclass[a4paper, 12pt]{amsart}

\usepackage{amsmath}
\usepackage{amssymb}
\usepackage[T1]{fontenc}
\author[M. Niebrzydowski]{Mateusz Niebrzydowski}
\title{Wprowadzanie w tryb matematyczny}

\begin{document}

\maketitle

\section{Środowisko trybu matematycznego}
\subsection{Przykłady}

\(x\),
$y$,
\begin{math}
z
\end{math}

$$x$$

\[y\]

\begin{displaymath}
z
\end{displaymath}

\begin{equation}
w
\end{equation}

\begin{equation*}
v
\end{equation*}


\subsection{Zadanie}
Ułamek wewnątrz akapitu $\frac{\frac{1}{x+y}-1}{a+b+c}$ i w trybie eksponowanym:
$$\frac{\frac{1}{x+y}-1}{a+b+c}.$$
Inna możliwość wewnątrz akapitu: \begin{math}\frac{\frac{1}{x+y}-1}{a+b+c}\end{math}
\section{Symbole matematyczne}
\subsection{Zadania na wyszukiwanie}\
\subsubsection{Pierwiastki}

$\sqrt{x}, \sqrt{x}+3, \sqrt{x+3}, \sqrt[3]{\sqrt{x}}+7, \sqrt[3]{\sqrt{x}+7}$ albo
$$\sqrt[3]{\sqrt{x}+7}.$$

\subsubsection{Litery greckie}

$\alpha, \beta, \Gamma, \gamma \Delta, \delta, \varepsilon, \epsilon, \Phi, \phi, \varphi, \theta, \vartheta, ...$
$$B(x,y)=\frac{\Gamma(x)\Gamma(y)}{\Gamma(x+y)}.$$

\subsubsection{Indeksy górny i dolny}

$a_{5}, x^{3+y}, A_{n+1}^{i,j,k}, e^{i\pi}=-1,$
$$a_{1}x^{2}e^{-\alpha t}a^{3}_{ij}e^{x^{2}}=(e^{x})^{2}$$

\subsubsection{Symbole relacji}

$<, \leq, >, \geq, \neq, \subset, \subseteq, \supset, \supseteq, \in, \parallel, \nparallel, \notin$
$$A=\{1,x\} \subseteq B = \{1,7,x,(b_{i})_{i \in I}\} \neq C = \{1,7,\{x\}, (b_{i})_{i \in I}\}$$

\subsubsection{Zbiory liczbowe}

$\mathbb{N}, \mathbb{Z}, \mathbb{Q}, \mathbb{C}$
$$\mathbb{N} \subset \mathbb{Z} \subset \mathbb{Q} \subset \mathbb{C}$$

\subsubsection {Funkcje}

$\cos{x}, \sin{x}, \lg{x}$
$$\cos(2\theta)=\cos^{2}(\theta)-\sin^{2}(\theta)$$

\subsubsection{Logika i teoria mnogości}
$\exists, \nexists, \forall, \neg, \land, \lor$


\subsubsection{Sumy, iloczyny i całki}

\subsection{Matematyczne kroje pisma}

\subsection{Zadanie. Funkcja Eulera}



\end{document}